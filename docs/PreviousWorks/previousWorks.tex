%\documentclass[openany]{book}
\documentclass[12pt]{article}
\usepackage[utf8]{inputenc}
\usepackage{amssymb} %for fancy L
\usepackage{calrsfs} %for fancy L
\usepackage{cancel}
\usepackage{tabularx}
\usepackage[hyphens]{url}
\usepackage{booktabs}
\usepackage{graphicx}
\usepackage[titletoc,title]{appendix}
\usepackage{subfig}
\DeclareMathAlphabet{\pazocal}{OMS}{zplm}{m}{n} %for fancy L
\usepackage{epsfig, float,array,tabu,longtable,}
\usepackage{hyperref,wrapfig}
\usepackage{enumerate}
\usepackage{graphicx,psfrag}
\usepackage{cite}
\usepackage{sectsty}
\usepackage{epstopdf}
\usepackage{amsmath,esint, setspace, fancyhdr, amsfonts, bookmark, blindtext}
\usepackage[normalem]{ulem}
\usepackage{tikz}
\usepackage{rotating}
\usepackage[americanvoltages,fulldiodes,siunitx]{circuitikz}
\usepackage{stackengine}
\usetikzlibrary{matrix}
\usepackage{multirow}
\usepackage{multicol}
\usetikzlibrary{shapes,backgrounds,patterns}
\usetikzlibrary{mindmap,trees,decorations.markings}
\usetikzlibrary{quotes,angles}
\usepackage{verbatim}
\renewcommand{\baselinestretch}{1}
\setlength{\textheight}{8in}
\setlength{\textwidth}{6.5in}
\setlength{\headheight}{0in}
\setlength{\headsep}{0.25in}
\usepackage{graphicx}
\setlength{\topmargin}{0in}
\setlength{\oddsidemargin}{0in}
\setlength{\evensidemargin}{0in}
\setlength{\parindent}{.3in}
\usepackage{listings}
\usepackage{color} %red, green, blue, yellow, cyan, magenta, black, white
\definecolor{mygreen}{RGB}{28,172,0} % color values Red, Green, Blue
\definecolor{mylilas}{RGB}{170,55,241}
\doublespacing
\begin{document}


\begin{titlepage}

\newcommand{\HRule}{\rule{\linewidth}{0.5mm}} % Defines a new command for the horizontal lines, change thickness here

\center % Center everything on the page
 
%---------------------------------------------------------
%	HEADING SECTIONS
%---------------------------------------------------------

\textsc{\LARGE Colorado School of Mines}\\[1.5cm] % Name of your university/college
\textsc{\Large CSCI 444}\\[0.5cm] % Major heading such as course name
\textsc{\large Advanced Computer Graphics}\\[0.5cm] % Minor heading such as course title

%---------------------------------------------------------
%	TITLE SECTION
%---------------------------------------------------------

\HRule \\[0.6cm]
{ \huge \bfseries Existing Work Survey}\\[0.4cm] % Title of your document
\HRule \\[1.0cm]
 
%---------------------------------------------------------
%	AUTHOR SECTION
%---------------------------------------------------------

\begin{minipage}{0.4\textwidth}
    \begin{flushleft}
        \emph{Author:}  
        \medskip
        {\textsc{\textbf{Robinson Merillat }}}   % Your name, bold and small caps    
    \end{flushleft}
\end{minipage}
\begin{minipage}{0.45\textwidth}
    \begin{flushright} \large
        \emph{Supervisor:}
        {\textsc{\textbf{Dr. Jeffery Paone }}} % Supervisor's Name
    \end{flushright}
\end{minipage}\\[1cm]

%---------------------------------------------------------
%	DATE SECTION
%---------------------------------------------------------
\begin{center}
{\large \today}
\end{center}
 % Date, change the \today to a set date if you want to be precise

%---------------------------------------------------------
%	LOGO SECTION
%---------------------------------------------------------
%\vfill
\newcommand*{\plogo}{\includegraphics[width=0.25\textwidth]{../project_proposal/imgs/mines.png}}

\plogo\\[1cm] % Include a department/university logo - this will require the graphicx package
 
%---------------------------------------------------------

\vfill % Fill the rest of the page with whitespace
\end{titlepage}

\newpage

% SUMMARY %%%%%%%%%%%%%%%%%%%%%%%%%%%%%%%%%%
\section{Previous Works}
\subsection{Designer Worlds: Procedural Generation of Infinite Terrain from Real-World Elevation Data}

This research paper delves into sending real world geographical elevation data into a variation of the perlin 
noise algorithm called value noise. The geographical data used is from the United States Geological Survey of Utah.
The value noise algorithm differs from the noise algorithm in that it chooses a randow height to interpolate between 
as opposed to the gradients. Additionally value noise isn't contrained to be zero at grid points. It has a small speedup 
over perlin noise, however this is only visible at higher dimensions.

1. \href{http://jcgt.org/published/0003/01/04/}{Link}

\subsection{Realtime Procedural Terrain Generation}

This paper provides an overview of erosion techniques used in games for terrain synthesis. It also goes into 
detail on how basic perlin noise is inferrior to voronoi diragrams. The 2D texture is then perturbed. Afterwards, 
seperate erosion techniques can be used such as thermal erosion, Hydraulic erosion, or an algorithm that the writers 
propose that combines the advantages of the two previously mentioned algorithms.

2. \href{https://pdfs.semanticscholar.org/5961/c577478f21707dad53905362e0ec4e6ec644.pdf}{Link}

\subsection{A Survey of Procedural Methods for Terrain Modelling}

This paper surveys multiple different generation technques such as height-mapping, generating rivers, 
oceans, and lakes, plant modeling and vegetation distrobution, Road networks, and urban environments.
One such technique for vegetation generation uses relative elevation, slope, and multi fractal noise 
to define ecosystems and the ground vegetation is generated at run-time. Another techniques ....

3. \href{http://www.cg.its.tudelft.nl/Publications-new/2009/SDGTB09a/SDGTB09a.pdf}{Link}

\subsection{GDC talk on Building worlds with Noise Gerneration in No Man's Sky}

This video clip from the Game Developer's Conference in March of 2017 discusses the techniques used by the 
developers of No Man's Sky to procedurally generate an entire univers in no more than 300MB of code, and 200GB of 
pre-developed assets. They create their own generation technique using a new noise method dubbed Uber Noise by 
the speaker Sean Murray. It combines several noise methods including Analytical derivative and domain warping.

4. \href{https://www.youtube.com/watch?v=SePDzis8HqY}{Link}

\subsection{Procedural Generation of Rock Piles using Aperiodic Tiling}

This paper discusses a tiling method for generating piles of rocks without any computationally demanding simulations. 
It also attempts to achieve a more realistic outcome than previous tiling methods. It utalizes a modified corner cube 
algorithm to generate a set of aperiodic (not periodic/ irregular) tiles. This also makes use of Varonoi polygons to 
enhance the control the shape of the rocks. The method that the writers discuss not only using ellipsoidal anisotropic distance
to generate the shape of the rocks as voxels, but also take into account rock erosion.

5. \href{https://perso.liris.cnrs.fr/adrien.peytavie/publications/articles/PG2009_ProceduralGenerationOfRockPilesUsingAperiodicTiling.pdf}{Link}

\section{Sources}

\begin{enumerate}
    \item Ian Parberry, Designer Worlds: Procedural Generation of Infinite Terrain from Real-World Elevation Data, Journal of Computer Graphics Techniques (JCGT), vol. 3, no. 1, 74-85, 2014
    \item Olsen, J. (2018). Realtime Synthesis of Eroded Fractal Terrain for Use in Computer Games. [ebook] Available at: https://pdfs.semanticscholar.org/5961/c577478f21707dad53905362e0ec4e6ec644.pdf [Accessed 23 Mar. 2018]. 
    \item Smelik, R., Tutenel, T., Bidarra, R. and Benes, B. (2014). A Survey on Procedural Modelling for Virtual Worlds. Computer Graphics Forum, 33(6), pp.31-50.
    \item Sean Murray GDC-Talk (2017) | Building Worlds with Noise Generation | No Man's Sky. (2017). [video] Game Developer's Conference
    \item Peytavie, A., Galin, E., Grosjean, J. and Merillou, S. (2009). Procedural Generation of Rock Piles using Aperiodic Tiling. Computer Graphics Forum, 28(7), pp.1801-1809
\end{enumerate}

\end{document}